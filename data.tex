\section*{The Data Set}

Over the last two years, Runestone Interactive has been gathering information about the websites’ users. Every time a user interacts with the site, a row is added to a table that captures specifics about the event, including the owner of the action; its timestamp, the course the user is associated with, and the event that occurred (i.e. a page load, or the answering of a multiple choice question) along with its result (i.e. whether or not the student answered the multiple choice question correctly). None of the user’s sensitive information is stored in the database. As the database has grown, this information has become more useful for analyzing student activity.   
The online textbook has experienced a huge amount of use by universities, secondary schools, and third party users, causing the database to grow rapidly. After only two years, the database has logged over 17 million actions by students and hobbyists. Universities that use the course include the University of Toronto, the University of Kentucky, and the University of Georgia.
In order to do analysis on the information gathered by the website, it had to be cleaned. There were three main groups that were removed: non-students, corrupt/miscellaneous rows, and outliers. Deciding who was a student, and who was not, was based on many different criteria. It was clear from the beginning that users who only visited to website once shouldn’t be considered students. However, other non-students included all of the course instructors, user IDs that were shared by entire classes of students, and users who didn’t log enough activity to allow for reasonable analysis. 
The corrupt/miscellaneous group of removed data takes out data that includes; pages and interactive elements that don’t appear in the official list provided by the textbook, users that have no username, or courses that are not associated with either of the two analyzed books.  
Outliers were calculated by two different parameters: clicks and duration. The clicks measurement is a count of how many events each user logged on the website, while duration measures how long a user is active on the site. For each parameter, the average and standard deviation are found for all users. Any user that has clicks or duration outside of three standard deviations from the mean is removed as outliers. In order to accurately calculate the outliers, they are found after all database cleaning has taken place. All in all, even after almost half of the database was removed, we were left with 9 million entries in a cleaned dataset ready for analysis. 
In order to provide a level of confidentiality to the websites users, all user and course names have been anonymized. Both registered and unregistered users are assigned an ID number that all of their information is connected to. Courses are also given ID numbers, along with a specification of what the course type is, whether it is a high school course, a college course, or one offered in an informal academic setting.  

While single time users were removed from the dataset because they disrupt the analysis of those students who are using the online textbook as a course, they do provide interesting insight. By finding the pages with the most hits from single time users, we gained an insight into some of the topics students sought  for online help, indicating topics they struggle with. We found the chapters on sorting algorithms and the Stack data structure to be the most common hits by single time users.  Another interesting statistic we looked at was the single time user activity over time. Not surprisingly we saw an overall increase in the website’s activity the longer it was up, but we also found a very uncharacteristic spike in the graph. We found that on Dec 13, 2011 there was a spike in the number of google hits unaffiliated with any courses.  There was almost %250 more hits on that day than any other day in the graph. Eventually we found that on that day a link to the website had been shared on Reddit, causing thousands of new users to explore the site. While this was of no interest to our particular research, it further showed the impacts of social media. 
