\section*{Related Work}


Gaining a better understanding of how students interact with and react to their programming environment has long been an important part of teaching novice programmers.  Since programming involves learning a programming language and then using it correctly and effectively, aspects of programming language system implementation are often linked to student success.  For example, the role of syntax errors as a cause of frustration and a barrier to learning has been of interest for many years [A,B].  Without syntactic correctness, students cannot move on to more logically based debugging since programs will not execute.  

Although most of these efforts have involved compiler errors from languages such as Java, other languages have also been used as the basis for studying student response to errors.  For example, Marceau et al (M) used a Scheme-based environment to develop a language independent rubric for classifying many types of student errors as a way to understand the effectiveness of error messages.  Since error messages are one of the first points of contact for a student learning to program, one would assume that information they provide is extremely important.  Denny [C] enhanced the error messages by including examples of how the error occurs and possible ideas for what can be done to fix the error.  Interestingly, even with these enhancements, students showed no significant improvement when it came to resolving the error.

In order to study student behavior with respect to resolving errors, it is useful to have comprehensive datasets from students carrying out the same task.  Projects such as BlueJ have allowed researchers to gather large amounts of data pertaining to the behavior of novice programmers [I J].  The BlueJ Blackbox project [K] has begun to collect data related to IDE use and editing history.  This rich dataset is also being made available to other researchers.

Of particular interest for this paper is the notion that Python is considered an excellent language for use in introductory computer science courses [D,E,F,G].  Furthermore, it has even been suggested that it may now be the most popular programming language for introductory teaching [N]. Still, like all programming language environments, Python error messages are not always easy for novice programmers. 

Recently, Miller and Ranum [O,P] have created Runestone Interactive as a new and unique vision for the creation and use of electronic textbooks. The Runestone project provides tools for creating active components and content in the context of an open source authoring system.  In addition, since the courseware is offered online, perpetual data gathering is possible giving rise to the Runestone Data Set.


