\section*{Introduction}

Anyone who teaches introductory computer science knows that learning to solve prolems with a new programming language can be quite difficult.  Not only do students have to begin thinking algorithmically but they also must learn to navigate a complex environment of syntax and semantics.  Much of the early frustration with programming can often be traced to difficulty understanding and knowing how to respond to error messages that are provided by the system.  Often, these messages can be cryptic and even misleading as students try to fix errors and make progress toward a working program.

In this paper, we consider error messages that are seen by beginning Python programmers.  By using a large, web enabled data set, we are able to classify errors by number and difficulty by looking at the occurrence patterns over time.  In addition, by virtue of having control over the Python runtime system, we are able to modify the error messages in order to provide possibly better information to the programmer, thereby helping to alleviate some of the time that is spent fixing these types of programming errors.

